% This text is proprietary.
% It's a part of presentation made by myself.
% It may not used commercial.
% The noncommercial use such as private and study is free
% Sep. 2005 
% Author: Sascha Frank 
% University Freiburg 
% www.informatik.uni-freiburg.de/~frank/
% additional use of \usepackage{beamerthemesplit}

\documentclass{beamer}

\begin{document}
\title{Bases llenas con dos outs, Pit\'agoras al bate} 
\author{Alejandro Crema} 
\date{Noviembre de 2018\\Happy Hour ASOVAC\\
Aula 35\\
Caracas, Venezuela\\
$\;$\\
Con la asesor\'ia de Fernando Crema} 

\frame{\titlepage} 
\frame{\frametitle{Contenido}\tableofcontents} 

\section{Porqu\'e hay tanta estad\'istica en el beisbol?} 

\frame{\frametitle{Porqu\'e hay tanta estad\'istica en el beisbol?} 

$\bullet$ Secuencia de estados y eventos que producen la transici\'on de un estado a otro. Se acumulan carreras anotadas. \break
	
Ejemplo:

\[s_1= ((-,-,-),0),\;e_1 = \mbox{Out43},\;s_2 = ((-,-,-),1)\]
\[s_2 = ((-,-,-),1),\;e_2 =2B,\;s_3 = ((-,*,-),1)\]
\[s_3 = ((-,*,-),1),\;e_3 = 1B + CA,\;s_4 =((*,-,-),1)\]
\[s_4 =((*,-,-),1),\;e_4=\mbox{out64 y out43},\;((X,X,X),3)\]  
   
$\bullet$ Se anota TODO lo que ocurre en relaci\'on a estados y eventos (sin comentarios cualitativos). \break
	
$\bullet$ Es muy sencillo contar ocurrencias de eventos y estados e  inventar m\'etricas (variables) para jugadores y equipos. 

}

\section{Ejemplos de variables tradicionales} 

\frame{\frametitle{Ejemplos de variables tradicionales}

$\bullet$ Total de eventos por jugador, se suman, se calculan relaciones, promedios, promedios ponderados, ...

\[BB,BBI,GP,TS,SF,1B,2B,3B,HR,BR,OR,CI,CA,AP,\cdots\]

\[H=1B+2B+3B+HR,\;AB=AP-BB-GP-TS-SF,\cdots\]

\[AVE=\frac{H}{AB},\;SLG = \frac{1B + 2x2B + 3x3B + 4xHR}{AB}\]

\[OBP = \frac{H + BB + GP}{AP-TS},\cdots\]

$\bullet$ Todas tienen un significado preciso en el contexto del beisbol .
}

\section{Sabermetr\'ia} 

\frame{\frametitle{Sabermetr\'ia}  

$\bullet$ SABR: Society for American Baseball Research (1971). \break

$\bullet$ METRICS: M\'etricas. \break 

$\bullet$ SABR + METRICS = SAB{\bf E}RMETRICS. \break

$\bullet$ Sabermetr\'ia: desarrollo EMPIRICO de variables sobre beisbol. Se va mucho m\'as all\'a de conteo y promedios ponderados tradicionales. \break 

$\bullet$ Algunos de los pioneros: Craig R. Wright (empleado de la MLB, el primer Sabermetrician, 1981), Davey Johnson (pelotero, manager y matem\'atico, 1965), Bill James (escritor sobre beisbol, 1980), Billy Bean (gerente de los Atl\'eticos de Oakland, 2001).

}

\section{Para qu\'e?}

\frame{\frametitle{Para qu\'e?} 

Por siempre:\break

$\bullet$ Para divertirnos, para discutir, para que hagamos click en p\'ags WEB (FanGraphs, Baseball References, MLB, LVBP), para hacer un Happy Hour de Asovac... \break

$\bullet$ Para estimar el valor de un jugador. \break

En la era saberm\'etrica:\break 

$\bullet$ Para pronosticar carreras anotadas, recibidas y victorias.\break 

$\bullet$ Para estimar el valor de un jugador en funci\'on de su aporte a las carreras anotadas, a las recibidas y a las victorias.\break

$\bullet$ Para dise\~nar equipos y los lineups de los mismos.\break
}


\section{Ejemplos de variables saberm\'etricas} 

\frame{\frametitle{Ejemplos de variables saberm\'etricas}

$\bullet$ Variables saberm\'etricas: funciones de funciones de .. de variables tradicionales:

\[OPS = OBP + SLG\]

\[wOBA = \frac{1.95 HR + 1.55 3B - \cdots  + \dots + 0.72 (BB-BBI)}{AP-TS-BBI}\]

\[\mbox{pythagorean variable} =  pyt = \frac{CA^\gamma}{CA^\gamma + CR^\gamma}\]

\[WAR,\;BABIP,\;RAR,\cdots\]

$\bullet$ Para entender lo que intentan medir debe examinarse la construcci\'on de los par\'ametros que las definen. Pronostican carreras y victorias? Eval\'uan peloteros?

}

\section{Construcci\'on del $wOBA$}

\frame{\frametitle{Construcci\'on del $wOBA$ (The Book: Tom M Tango, Mitchell G Lichtman, Andrew E Dolphin (2007) Ed. Potomac)} 

$\bullet$ Tango dice que The Book ES {\it el librito}.\break

$\bullet$ El libro tiene errores conceptuales asombrosos (de matem\'aticas, de estad\'istica, de teor\'ia de probabilidades).\break 

$\bullet$ El libro es extraordinario (han podido obviar los basamentos matem\'aticos que en realidad no usan). Construyen paso a paso una de las variables saberm\'etricas mas reconocidas: el $wOBA$. 

}


\frame{\frametitle{$wOBA$}

Recordemos el $SLG$:

\[SLG = \frac{1B + 2x2B + 3x3B + 4xHR}{AB}\]

$\bullet$ En el $SLG$ solo aparecen 4 eventos. Los eventos siempre valen lo mismo sin importar los estados antes y despu\'es del evento.\break 

Propuesta de Tango:

\[E = \lbrace BB,BBI,GP,1B,\cdots,SO,out,\cdots \rbrace,\;\; \vert E \vert = 20\]

\[S = \lbrace ((-,-,-),0), \cdots , ((*,*,*),2),\;\vert S \vert = 24\]

\[v(e,sI,sF):\mbox{valor del evento};e\;\mbox{al pasar del estado}\;sI\; \mbox{al estado}\;sF\]

\[ce(s):\mbox{carreras anotadas en promedio partir de s}\]

}




\frame{\frametitle{wOBA}

Ejemplos:\break

\[\bullet\;sI = ((-,*,-),0),\;e=1B+CA,\;\;sF=((*,-,-),0)\]

\[v(e,sI,sF) = 1 + ce(sF) - ce(sI) = 1 + 0.953 - 1.189 = 0.764\] 

\[\bullet sI = ((-,*,-),0),\;e=1B+CA+out4,\; sF =  ((-,-,-),1)\]

\[v(e,sI,sF) = 1 + ce(sF) - ce(sI) = 1 + 0.297 - 1.189 = 0.108\] 

\[\bullet sI = ((*,*,*),2),\;e=HR+4*CA,\;sF = ((-,-,-),0)\] 

\[v(e,sI,sF) = 4 + ce(sF) - ce(sI) = 4 + 0.117 - 0.815 = 3.302\] 

}

\frame{\frametitle{Par\'entesis}

Para los defensores de los toques de sacrificio:

\[ce((*,-,-),0)=0.953,\;\;ce((-,*,-),1)=0.725\]
\[ce((*,*,0))=1.573,\;\;ce((-,*,*))=1.467\]

}



\frame{\frametitle{$wOBA$}

\[wOBAjug = \frac{\sum_{e \in E} \sum_{sI \in S} \sum_{sF \in S} v(e,sI,sF) njug(e,sI,sF)}{APjug-BBIjug-TSjug}\]

\[= \cdots =\sum_{e \in E} vjug(e) propjug(e)\]

Tango (2007): No se conoce $njug(e,sI,sF)$ pero s\'i conoce $nliga(e,sI,sF)$ y por supuesto $propliga(e,sI,sF)$. Eso es falso hoy en d\'ia. Pueden calcular $wOBAjug$ correctamente. El $wOBA$ original es:

\[wOBA = \sum_{e \in E} vliga(e) propjug(e)\]
}




\frame{\frametitle{$wOBA(2018)$ seg\'un FanGraphs}

\[wOBA = \]

\[\frac{0.69BBnI + 0.72GP + 0.89(1B) + 1.27(2B) + 1.62(3B) + 2.1HR}{AP- BBI - TS}\]

$\bullet$ Tango: para los defensores del $OPS$

\[wOBA \approx \frac{2*OBP + SLG}{3}\;\;\mbox{vs}\;\; OPS = OBP + SLG\]


$\bullet$ $wOBA$: weighted On Base Average. \break

$\bullet$ Hay factores de correcci\'on por stadiums (no se batea igual en todas partes, usan Estad\'istica y F\'isica para corregir)!  
}

\frame{\frametitle{C\'omo usar y mejorar el wOBA y otras variables saberm\'etricas}

$\bullet$ A partir del $wOBA$ se definen otras variables.\break

$\bullet$ Usar $njug(e,sI,sF)$ y no $nliga(e,sI,sF)$. Lo usan?. \break

$\bullet$ El $wOBA$ claramente evalua jugadores. Mientras m\'as alto mejor.
\break

$\bullet$ No hay que exagerar: {\it $wOBA$ is a statistic that uses linear weights (read: math) to determine {\bf exactly} how valuable each offensive outcome truly is)}\break


$\bullet$ Bill James: calcular $v(e,sI,sF)$ seg\'un el contexto. Puede definirse el $wOBA$ y todas las variables saberm\'etricas para situaciones cr\'iticas.\break

$\bullet$ Sirve para pronosticar carreras y  victorias? Conexi\'on con Programaci\'on Matem\'atica.   
}

\section{Programaci\'on matem\'atica para la selecci\'on de variables}

\frame{\frametitle{Programaci\'on matem\'atica para la selecci\'on de variables} 

$\bullet$ Modelo para pronosticar $Y$ ($CA$, $CR$, $JG$) usando las variables $X_i$ ($H,1B,\cdots,BABIP,WAR$).\break

$\bullet$ Datos: Para cada variable $X_i$ y cada equipo $j$ en la muestra hist\'orica. Valores de las variables $X^{(j)}_i$ y $Y_j$.\break

$\bullet$ El Modelo lineal en $X_i$:

\[Y = \alpha_0 + \sum_{i=1}^n \alpha_i X_i + \epsilon \]

}

\frame{\frametitle{Selecci\'on de variables} 

C\'omo hallar el mejor modelo que use solamente $k$ variables?

\[P(k)\;\;min\;\;\sum_{j=1}^n \vert \epsilon_j \vert\;\;s.a:\]
\[\epsilon_j = Y_j - (\alpha_0 + \sum_{i=1}^n \alpha_i X^j_i)\]
\[z_i = 0 \implies \alpha_i = 0\]
\[\sum_{i=1}^n z_i= k\]
\[\alpha_0,\alpha_i \in \Re^n,\epsilon_j \in \Re,\;\;z_i \in \lbrace 0,1 \rbrace\]
}

\frame{\frametitle{Selecci\'on de variables. A.Crema y F.Crema (2016)} 

$\bullet$ $P(k)$ se reescribe adecuadamente para que se pueda, en teor\'ia, resolver con alg\'un software de Programaci\'on Entera apropiado (por ejemplo CPLEX). Pero, si $n$ es grande y $k$ est\'a cerca de $\frac{n}{2}$ el tiempo de CPU puede ser inadmisible.\break

$\bullet$ Un buen modelo usando $k$ variables puede construirse con un buen modelo usando $k-1$ variables a\~nadiendo y eliminando un n\'umero controlado de variables. \break

$\bullet$ Se resuelve $P(1),P(2),\cdots,P(k_1)$. Luego se usan B\'usquedas Locales para hallar buenos modelos para $k_1+1,k_1+2,\cdots,k_2$ para luego resolver $P(k_2+1),\cdots,P(n)$   

}

\frame{\frametitle{Selecci\'on de variables}

\begin{table} 
 \caption{Modelos de pron\'ostico de Carreras. Busquedas locales sucesivas 1389 equipos de MLB (1967-2914) (kmax=7)}
 \label{tab:4}
 \begin{tabular}{lllll}
 \hline\noalign{\smallkip}
 Modelo & k & error promedio & maxerror & Tiempo (seg.)\\
 \noalign{\smallskip}\hline\noalign{\smallskip}
  \mbox{Atomos} &  13   & 26.68 &    296 & 20\\
  \mbox{At+Ave} &  19   & 26.31 &    282 & 64\\ 
  \mbox{Sabr}   &   4   & 16.99 &     75 &  7\\
                &   8   & 16.92 &     77 &   \\
  \mbox{Ave+Sabr}&   6   & 16.82 &    74 &  23\\
                 &  14   & 16.76 &    75 &\\
  \mbox{Todas}   &  27   & 15.69 &    88 &  514 \\
  \noalign{\smallskip}\hline
 \end{tabular}\\
 \end{table}

}

\frame{\frametitle{Selecci\'on de variables}

\begin{table} 
 \caption{Modelos de pron\'ostico de Porcentaje de Victorias. Busquedas locales sucesivas 1389 equipos de MLB (1967-2914) (kmax=5)}
 \label{tab:4}
 \begin{tabular}{lllll}
 \hline\noalign{\smallkip}
 Modelo & k & error promedio & maxerror & Tiempo (seg.)\\
 \noalign{\smallskip}\hline\noalign{\smallskip}
  \mbox{Atomos} &  11   & 2.5  &    10.74 & 78\\
                &  26   & 2.4  &    10.08 &   \\
  \mbox{At+Ave} &  26   & 1.9 &     8.70 & 437\\ 
                &  33   & 1.89 &    8.71 &    \\ 
  \mbox{Sabr}   &  24   & 1.85 &    8.32 & 1610 (12600) \\
                &  40   & 1.83 &    8.38 &           \\
    
  \noalign{\smallskip}\hline
 \end{tabular}\\
 \end{table}

}

\section{Construcci\'on de la variable Pitag\'orica: $pyt$}

\frame{\frametitle{Construcci\'on de la variable Pitag\'orica: $pyt$ (Baseball abstracts: Bill James (1983) Ed. Ballantine). La idea primitiva: un modelo elemental v\'alido.} 

Se gana si $CA > CR$. Se define la Calidad de los equipos:

\[Cal_1 = \frac{CA_1}{CR_1},\;\;Cal_2 = \frac{CA_2}{CR_2} = \frac{CR_1}{CA_1}\]

Modelo probabil\'istico elemental:

\[ProbGana_1 = \frac{Cal_1}{Cal_1+Cal_2},\;\;ProbGana_2 = \frac{Cal_2}{Cal_1+Cal_2}\]

con lo cual:

\[ProbGana_1 = \frac{\frac{CA_1}{CR_1}}{ \frac{CA_1}{CR_1}   +   \frac{CR_1}{CA_1}} = \frac{CA_1^2}{CA_1^2 + CR_1^2}\]

}


\frame{\frametitle{Construcci\'on de la variable Pitag\'orica. La pr\'actica}

\[pyt = \frac{CA^2}{CA^2 + CR^2},\;pyt = \frac{CA^\gamma}{CA^\gamma + CR^\gamma}\]

\[\gamma\;\;\mbox{se ajusta con los datos de una liga usando m\'etodos estad\'isticos}\]

\[min \sum_{i=1}^n (Prop_i-pyt_i)^2\]

$\bullet$ Asumen el mismo $\gamma$ para toda la liga.\break 

$\bullet$ No toman en cuenta el calendario restante. Las proyecciones podr\'ian ser incompatibles. Eso puede arreglarse a\~nadiendo restricciones al problema.

} 

\frame{\frametitle{Validaci\'on experimental-anal\'itica de la variable  pitag\'orica (Steven J. Miller: A Derivation of the Pytahgorean won-loss formula in Baseball. Chance 20 (1), 40-48 (2007))} 

$\bullet$ Miller valida experimentalmente que $CA$ y $CR$ pueden ajustarse con  Variables Weibull independientes con par\'ametros $\alpha_A,\gamma$ y $\alpha_R,\gamma$ respectivamente. \break 

$\bullet$ Lo que sigue es anal\'itico (luego de algunas cuentas muy complicadas):

\[\mbox{Prob de ganar} =  \mbox{Prob} \lbrace CA > CR \rbrace = \frac{\mu_A^\gamma}{\mu_A^\gamma + \mu_R^\gamma}\]

con $\mu_A$ y $\mu_R$ los valores esperados (promedios) de $CA$ y $CR$

$\bullet$ $\gamma$ deber\'ia depender del equipo y de si se trata de $CA$ o de $CR$. En la pr\'actica se busca un solo $\gamma$. 
 
}

\frame{\frametitle{Validaci\'on experimental-anal\'itica de la f\'ormula pitag\'orica} 

\[Prob \lbrace CA > CR \rbrace = \frac{\mu_A^\gamma}{\mu_A^\gamma + \mu_R^\gamma}\] con lo cual en la pr\'actica:

\[Prob(ganar) = pyt = \frac{CA^\gamma}{CA^\gamma + CR^\gamma}\] 
 


}



\frame{\frametitle{Para qu\'e sirve la f\'ormula pitag\'orica} 

\[\mbox{A mitad de temporada tenemos}\;\;prop,CA\;\;\mbox{y}\;\;CR\;\;\mbox{para un equipo}\] 

Calculamos

\[pyt = \frac{CA^\gamma}{CA^\gamma + CR^\gamma}\]

\[\mbox{y comparamos}\mbox{\;}prop\;\mbox{con}\;pyt\] 

Ya que el ajuste es bueno podemos concentrarnos en desarrollar modelos para proyectar $CA$ (con un modelo de la ofensiva)  y $CR$ (con un modelo para la defensiva y el pitcheo) para luego proyectar Prob de ganar usando la f\'ormula pitag\'orica.


}

\frame{\frametitle{$pyt$ para MLB y LVBP} 

$\bullet$ Se ajust\'o el modelo para la MLB con 1389 equipos desde 1967 hasta 2014.

\[\gamma = 1.85,\;\;\mbox{error promedio} = 0.0198,\;\mbox{maxerror} = 0.0895\]

$\bullet$ idem para la LVBP con 32 equipos desde 2015 hasta el domingo 18 de Noviembre de 2018. 

\[\gamma = 1.29,\;\;\mbox{error promedio} = 0.0372,\;\mbox{maxerror} = 0.1400\]


}

\frame{\frametitle{$pyt$ para la temporada 2018} 

$\bullet$ Se ajust\'o el modelo para la LVBP con 8 equipos (temporada 2018 hasta el domingo 18 de noviembre) 

\[\gamma = 1.4784,\;\;\mbox{error promedio} = 0.0323,\;\mbox{maxerror} = 0.0564\]

}

\frame{\frametitle{Proyecci\'on temporada 2018}

\begin{table} 
 \caption{Proyecci\'on temporada 2018}
 \label{tab:4}
 \begin{tabular}{lllllll}
 \hline\noalign{\smallkip}
 Equipo & G & P & prop & pyt & G & G \\
 \noalign{\smallskip}\hline\noalign{\smallskip}
 \mbox{Cardenales} &  25 & 12 & .636 & .597 & 38.9 & 39\\
 \mbox{Navegantes} &  18 & 14 & .563 & .526 & 34.3 & 34\\
 \mbox{Leones}     &  16 & 15 & .516 & .510 & 32.3 & {\bf 32}\\
 \mbox{Bravos}     &  17 & 16 & .515 & .548 & 33.4 & {\bf 33-34}\\
 \mbox{Aguilas}    &  15 & 17 & .469 & .514 & 30.9 & 31\\
 \mbox{Tigres}     &  14 & 16 & .467 & .453 & 28.9 & 29\\
 \mbox{Tiburones}  &  13 & 18 & .419 & .383 & 25.2 & {\bf 25}\\
 \mbox{Caribes}    &  13 & 19 & .406 & .456 & 27.1 & {\bf 27}\\
 \noalign{\smallskip}\hline
 \end{tabular}\\
 \end{table}
 
 $\bullet$ Con Bravos 34: 251 Ganados (Temporada tiene 252 juegos).

}


\end{document}



% % % % % % % % % % % % % % % % % % % % % % % % % % % % % % % % % % % % %
% % % % % % % % % % % % % % % % % % % % % % % % % % % % % % % % % % % %

\frame{\frametitle{Un par\'entesis}

$\bullet$ Lanzamiento de una moneda. Cara: c, Sello: s \break

$\bullet$ Modelo 1:

\[p(c) = \frac{1}{2},\;p(s) = \frac{1}{2}\]

\[p(c) \geq 0,\;p(s) \geq 0,\;p(c)+p(s)=1\]

$\bullet$ Modelo 2:

\[p(c) = \frac{3}{4},\;p(s) = \frac{1}{4}\]

\[p(c) \geq 0,\;p(s) \geq 0,\;p(c)+p(s)=1\]

$\bullet$ Ambos son modelos probabil\'isticos v\'alidos. Cu\'al usar?
}

\frame{\frametitle{Un par\'entesis}

$\bullet$ Disponemos de $1000$ datos. $\lbrace c,c,s,c,\cdots, s,s,c,s \rbrace$ \break

$\bullet$ 523 c y 477 s.\break

Ajuste del modelo 1: 

\[\frac{(523-500)^2 + (477-500)^2}{1000} = \frac{1058}{1000} = 1.058\] 

Ajuste del modelo 2: 

\[\frac{(523-750)^2 + (477-250)^2}{1000} = \frac{103058}{1000} = 103.058\] 

$\bullet$ El modelo 1 ajusta mejor.
}

\frame{\frametitle{Modelo ingenuo para las probabilidades de ganar juegos}

$\bullet$ Una liga de beisbol con $n$ equipos. CA: carreras anotadas. CR: carreras recibidas. \break

$\bullet$ Queremos un modelo para predecir las victorias de los equipos por el resto de la temporada. \break

$\bullet$ Datos hasta el momento: $(CA_1,CR_1,Porc(G)_1), \cdots ,(CA_n,CR_n,Porc(G)_n)$ \break 

$\bullet$ Modelo ingenuo para las probabilidades de ganar y perder un juego por cada equipo.

\[p_i(ganar) = \frac{CA_i}{CA_i+CR_i},\;p_i(perder) = \frac{CR_i}{CA_i+CR_i}\]

\[Error = \frac{\sum_{i=1}^n (Porc(G)_i - 100*p_i(ganar))^2}{n}\]  

}



  

% % % % % % % % % % % % % % % % % % % % % % % % % % % % % % % % % % % % %
\frame{\frametitle{}
\begin{figure}[!ht]
    \centering

\psset{
  xunit=0.5,
  yunit=0.5,
  algebraic
}



\begin{pspicture}(-0.25,-0.25)(10.00,10.00)
  % Axes
  \psaxes[
    dx=2,
    dy=2,
    labels=none,
    ticklinestyle=dotted,
    tickwidth=0.5pt,
    xticksize=0 0.5,
    yticksize=0 1,
  ]{->}(0,0)(-0.05,-0.05)(10.5,10.5)[$n$,0][$q(n)$,90]
  \psaxes[
    dx=2,
    Dx=2,
    dy=2,
    Dy=2
  ](0,0)(-0.05,-0.05)(10.00,10.00)
  % Functions.
  %\psplot[linecolor=red]{0}{1}{200*x+500}
  %\rput(.9,500){$y$}
  %\psplot[linecolor=black,linewidth=0.055]{0}{0.5}{0}
  \psdots[dotsize=3.0pt 0](0,2)(1,3)(2,5) (3,7) (4,6) (5,10)(6,5) (7,3) (8,4) (9,2) (10,1)  
  %\psplot[linecolor=black,linewidth=0.0155]{0.0}{10}{x}
  %\psplot[linecolor=blue,linestyle=dashed]{0}{1}{0.5*x}
  %\rput(.5,.375){$f^+(\theta)$}
  %\rput(.8,.125){$p\theta$} 
  %\psplot[linecolor=red,linestyle=dashed]{0}{1}{0.25*x}
\end{pspicture}

   \vspace*{5mm}     
   \caption{}
             
\end{figure}




}






